\documentclass[12pt,letterpaper]{article}
\usepackage{cite}
\usepackage{amsmath}
\usepackage{amsfonts}
\usepackage{array}
\usepackage{dsfont}
\usepackage{amssymb}
\usepackage{amsthm}
\usepackage{bbold}
\usepackage{fullpage}
\usepackage{mathtools}
\usepackage{enumitem}
\usepackage{mathrsfs}
\usepackage[margin=0.9 in]{geometry}
\usepackage{hyperref}
\usepackage{graphicx}
\usepackage{gensymb}
\usepackage{xcolor,colortbl}
\usepackage[format=plain,
labelfont={bf,it},
textfont={it}]{caption}
\usepackage{float}


\newcommand*{\SignatureAndDate}[1]{%
	\par\noindent\makebox[2.5in]{\hrulefill} \hfill\makebox[2.0in]{\hrulefill}%
	\par\noindent\makebox[2.5in][l]{#1}      \hfill\makebox[2.0in][l]{Date}%
}%
\newcolumntype{L}{>{\centering\arraybackslash}m{2cm}}
\newcolumntype{P}{>{\centering\arraybackslash}m{3cm}}
\newcolumntype{Q}{>{\centering\arraybackslash}m{4cm}}
\setlength{\parindent}{0em}

\allowdisplaybreaks

\newcommand{\R}{\mathds{R}}
\newcommand{\Z}{\mathds{Z}}
\newcommand{\Rplus}{\mathds{R}_{> 0}}
\newcommand{\Zplus}{\mathds{Z}_{\geq 0}}
\newcommand{\F}{\mathds{F}}
\newcommand{\N}{\mathds{N}}
\newcommand{\T}{\mathds{T}}
\newcommand{\s}{\mathds{S}}
\newcommand{\C}{\mathds{C}}
\newcommand{\CDFT}{\mathscr{F}_{CD}} %Fourier transform
\newcommand{\ip}[2]{\langle #1, #2\rangle}


\setlength{\parskip}{0.5em}


\makeatletter
\newsavebox\myboxA
\newsavebox\myboxB
\newlength\mylenA

\begin{document}
	
\section{Legal Considerations}
\subsection{Regulatory Considerations}

\subsubsection{Street Legality}

Many jurisdictions have laws and regulations pertaining to the operation of personal transport vehicles, both electric and human-powered. 
Toronto and Fredricton are among many Canadian cities which have laws prohibiting the use of skateboards on public streets. 3Chantale Nicole
With regards to motorized transport, California had a long-standing ban on motorized skateboards which was recently abolished in the passing of Assembly Bill 2054 in 2015. \cite{OCLaws} \cite{WSJLaws}
Recently, the emergence of two-wheeled, self balancing electric scooters prompted the development of new laws and regulations. 
These devices are banned from airlines in Canada and the USA and their usage is heavily restricted or banned in regions including Toronto, Vancouver, New York City, and the state of California. \cite{leetboard}
It is imperative that a personal electric transport vehicle be compliant with relevant regulations and laws in its area of operation.


\subsubsection{Intellectual Property}

In order to ensure that a personal electric transport system is commercially feasible, it is imperative that all the appropriate measures to protect intellectual property be undertaken. 
As the final product will involve the incorporation of mechanical control system to a pre-existing complex mechanical vehicle, the final product will qualify as a combinatory invention. 
A combinatory invention involves combining two or more pre-existing technologies in order to develop a new technology. 
Combinatory inventions are among the most common patents; between the years 1790 and 2010, 77\% of all patents granted involved the combination of at least two pre-existing technologies. \cite{COMB} 
A patent shall not be issued, under United States patent law, if \emph{the subject matter as a whole would have been obvious at the time the invention was made to a person having ordinary skill in the art to which said subject matter pertains. 
Patentability shall not be negatived by the manner in which the invention was made.}" \cite{NonObv} 
\\
\\
A 2007 United States Supreme Court ruling in "KSR International Co. v. Teleflex Inc." established legal precedent for "non-obviousness" in combinatory inventions. 
Teleflex Inc. sued KSR International Co. for patent infringement pertaining to a KSR product which Teleflex claimed infringed on its patent for an "Adjustable Pedal Assembly with Electronic Throttle Control."
KSR argued that Teleflex's claim was invalid as the action was obvious. 
The United States Supreme Court ruled in favour of KSR International Co.,  
ruling that Teleflex's invention was not patentable as the combination of the two adjustable pedal assembly and the electronic throttle control was indeed obvious. 
\\
\\
In developing novel methods of personal electric transport, it is imperative that the designers consider existing patent laws and cases in order to ensure that their products can be protected.


\subsection{Professional Practice Considerations}


\newpage
\bibliography{Bibliography}{}
\bibliographystyle{unsrt}

\end{document} 