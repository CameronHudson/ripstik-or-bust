\section{Engineering Considerations}
\subsection{Legal Considerations}

\subsubsection{Regulatory Considerations}

\paragraph{Street Legality}\mbox{}\\
Many jurisdictions have laws and regulations pertaining to the operation of personal transport vehicles, both electric and human-powered. 
Toronto and Fredricton are among many Canadian cities which have laws prohibiting the use of skateboards on public streets. \textbf{CITE? 3Chantale Nicole}
With regards to motorized transport, California had a long-standing ban on motorized skateboards which was recently abolished in the passing of Assembly Bill 2054 in 2015. \cite{OCLaws} \cite{WSJLaws}
Recently, the emergence of two-wheeled, self balancing electric scooters prompted the development of new laws and regulations. 
These devices are banned from airlines in Canada and the USA and their usage is heavily restricted or banned in regions including Toronto, Vancouver, New York City, and the state of California. \cite{leetboard}
It is imperative that a personal electric transport vehicle be compliant with relevant regulations and laws in its area of operation.


\paragraph{Intellectual Property}\mbox{}\\
In order to ensure that a personal electric transport system is commercially feasible, it is imperative that all the appropriate measures to protect intellectual property be undertaken. 
As the final product will involve the incorporation of mechanical control system to a pre-existing complex mechanical vehicle, the final product will qualify as a combinatory invention. 
A combinatory invention involves combining two or more pre-existing technologies in order to develop a new technology. 
Combinatory inventions are among the most common patents; between the years 1790 and 2010, 77\% of all patents granted involved the combination of at least two pre-existing technologies. \cite{COMB} 
A patent shall not be issued, under United States patent law, if \emph{the subject matter as a whole would have been obvious at the time the invention was made to a person having ordinary skill in the art to which said subject matter pertains. 
	Patentability shall not be negatived by the manner in which the invention was made.}" \cite{NonObv} 
\\
\\
A 2007 United States Supreme Court ruling in "KSR International Co. v. Teleflex Inc." established legal precedent for "non-obviousness" in combinatory inventions. 
Teleflex Inc. sued KSR International Co. for patent infringement pertaining to a KSR product which Teleflex claimed infringed on its patent for an "Adjustable Pedal Assembly with Electronic Throttle Control."
KSR argued that Teleflex's claim was invalid as the action was obvious. 
The United States Supreme Court ruled in favour of KSR International Co.,  
ruling that Teleflex's invention was not patentable as the combination of the two adjustable pedal assembly and the electronic throttle control was indeed obvious. 
\\
\\
In developing novel methods of personal electric transport, it is imperative that the designers consider existing patent laws and cases in order to ensure that their products can be protected.

\subsubsection{Professional Practice Considerations}
\subsection{Design Considerations}
%%Safety vs Power, cost vs power, battery life vs environmental considerations & power & cost, stability vs mobility, cost vs sustainability
\subsection{Ethics}
%Manufacturing, disposal, safety???, Marketing???
\subsection{Economic Analysis}
\subsubsection{Cost Breakdown}
In order to determine the economic feasibility of an electric transport device, it was compared to traditional gas and human powered transportation devices.
An analysis was conducted based on three different metrics; Initial cost, maintenance cost and fuel cost.
The electric powered devices used in the analysis were the OneWheel and Boosted board.
The Gas powered device used in the analysis was a 2017 Yamaha Zuma 50F.
The human powered device used in the analysis was a  Raleigh Furley bicycle.
\par
Initial cost looked at the upfront costs associated with purchasing each of the devices.
Maintenance cost looked at the necessary costs required for spare parts, and yearly upkeep of each device. 
Fuel costs looked at the cost required to power each of the devices.	
The economic analysis for each method was conducted over a span of five years, and the results can be seen in Table \ref{table:econ}.

\begin{table}[ht]
	\caption{Comparison of electric, gas, and human powered transportation devices over five years (all values in USD)}
	\centering
	\def\arraystretch{1.5}
	\begin{tabular}{|c| c| c| c| c|}
		\hline\hline
		\textbf{Method} & \textbf{Device}	& \textbf{Initial Cost (\$)} & \begin{tabular}{@{}c@{}} \textbf{Maintenance}\\ \textbf{Cost (\$)} \end{tabular} & \begin{tabular}{@{}c@{}} \textbf{Fuel} \\ \textbf{Cost (\$)} \end{tabular} \\ 
		\hline
		\multirow{2}{*}{Electric Powered} & OneWheel & 1,299 - 1,499 & 340 & 25 \\
		\cline{2-5}
		& Boosted Board & 1,299 - 1,499 & 359 & 25\\
		\hline
		Gas Powered & Yamaha Zuma 50F & 2,599 & 371.40 & 155\\ 
		\hline
		Human Powered & Raleigh Furley Bicycle & 980 & 969.90 & 0\\[0.1ex]	
		\hline
	\end{tabular}
	\label{table:econ}
\end{table}

The inital cost for the OneWheel was US\$1299.00 for the base model, and US\$1499.00 for the Plus model \cite{wheelcost}.	
The initial cost for the Boosted Board was US\$1299.00 for the base model, and US\$1499.00 for the plus model \cite{boardcost}.
The initial cost for the Yamaha Zuma 50F was US\$2,599.00 for the base model \cite{Yamaha}.
The initial cost for the Raleigh Furley Bicycle was US\$980.00 \cite{Raleigh}. 
The Raleigh was selected since it was ranked one of the top commuter bicycles for 2016 \cite{BikeMagazine}.
\par
The maintenance cost for the OneWheel totaled to US\$340.00, and consisted of one replacement charger (US\$125.00), and one tune up and reload pack (US\$215.00) \cite{wheelcost}.
The tune up and reload pack consisted of a 17 point inspection, motor, battery health, hardware and firmware assessment, new footpads, new bumpers, and a new tire \cite{wheelcost}.
The maintenance cost for the boosted board totaled to US\$359.00, and consisted of replacement parts for each of the key components on the board. This includes US\$105.00 for a full set of replacement wheels, US\$100.00 for a replacement remote, US\$79.00 for a replacement charger, US\$50.00 for a bearing service kit, and US\$25.00 for a motor belt service kit \cite{boardcost}.
The maintenance cost for the Yamaha Zuma 50F consisted of a maintenance kit with lube, oil filters, fuel filters,drain plugs, and a disposable funnel \cite{YamahaMaintenance}. 
This led to a cost of US\$74.28 per year, totaling to US\$371.40.
The maintenance cost for the Raleigh Furley bicycle consisted of a tune up and drivetrain clean, along with two new tires each year. 
The tune up and drivetrain clean cost US\$150.00 per year, totaling to US\$750.00 dollars \cite{bikerepair}. 
The replacement tires cost US\$44.00 per year, and totaled to US\$220.00 \cite{CanadianTire}.
\par
The fuel costs for the OneWheel and Boosted Board were treated equally, assuming that they both use the same battery type. With the specified battery, the average yearly charging cost is US\$5.00, assuming a travelling distance 2000 miles \cite{boostedkickstart}. 
This leads to a total cost of US\$25.00.
The fuel costs for the Yamaha Zuma 50F were calculated assuming the same yearly travel distance of 2000 miles.
The Yamaha has a fuel tank that can hold 1.2 Gallons of fuel, and a fuel efficiency of 132 miles per gallon \cite{Yamaha}.
The gas price was selected by assuming that the refills are occuring in New York State, leading to a price of US\$2.448 per gallon \cite{gasprice}.
With the information provided, the average fuel cost came to US\$31.00 per year, totaling to US\$155.00.
The fuel cost associated with a bicycle is zero, as it relies only on human force for operation.
\par
From a purely monetary standpoint, the electric powered transportation is the cheapest option.

\subsubsection{Usage Characteristics}
A few other factors need to be considered, such as the range of the device, top speed of the device, and environmental impact of using the device.
The range of the device will consider total distance until the fuel source runs out.
The top speed of the device will consider the maximum speed that the device can achieve in kilometers per hour.
The environmental impact of using the device will consider the emissions generated during vehicle operation.
The three factors were measured for each device and compiled in Table \ref{table:usage}.

\begin{table}[ht]
	\caption{Comparison of the range, top speed, and emissions for the transportation devices}
	\centering
	\def\arraystretch{1.3}
	\begin{tabular}{|c| c| c| c| c|}
		\hline\hline
		\textbf{Method} & \textbf{Device}	& \begin{tabular}{@{}c@{}} \textbf{Maximum} \\ \textbf{Range (km)} \end{tabular} & \begin{tabular}{@{}c@{}} \textbf{Top Speed}\\ \textbf{(km/h)}\end{tabular} & \textbf{Emissions}  \\ 
		\hline
		\multirow{2}{*}{Electric Powered} & OneWheel & 10-11 & 15-24 & 0 \\
		\cline{2-5}
		& Boosted Board & 10-19 & 35 & 0\\
		\hline
		Gas Powered & Yamaha Zuma 50F & 253 & 68 & \begin{tabular}{@{}c@{}} 1.0 g/km HC\\ 12 g/km $CO_{2}$\end{tabular}\\ 
		\hline
		Human Powered & Raleigh Furley Bicycle & 16 & 18-19 & 0\\[0.1ex]	
		\hline
	\end{tabular}
\label{table:usage}
\end{table}

The maximum range for the OneWheel is 10-11 km \cite{wheelcost}.
The maximum range for the Boosted Board is 10 km for the standard package, and 19 km for the plus package \cite{boardcost}.
Given a fuel efficiency of 132 miles per gallon and a tank size of 1.2 gallons, the maximum range of the Yamaha Zuma 50F is 253 kilometers.
The average bicycle commuter travels a distance of 16 kilometers to and from their destination \cite{BikePaper}, but this distance can vary based on the distance the rider needs to travel and their physical fitness level.
\par
The top speed of the OneWheel is between 15-24 kilometers per hour \cite{wheelcost}.
The top speed of the Boosted Board is 45 kilometers per hour \cite{boardcost}.
The top speed of the Yamaha Zuma 50F is 68 kilometers per hour \cite{Yamaha}.
The top speed of the Raleigh Furley Bicycle is dependant on the rider, but typical speed for bicycle riders are in the range of 18-19 kilometers per hour \cite{bikespeed}.
\par
The only device that generates any environmental waste is the Yamaha Zuma 50F. The average gas scooter emits approximately 1.0 gram/km of hydrocarbons and 12 grams/km of carbon dioxide \cite{emissions}.
\par
The key trade-off to consider when selecting a device is environmental impact versus convenience.
The Yamaha Zuma 50F has a significantly higher top speed and maximum range when compared to the other devices. 
However, it generates a significant amount of emissions. The electric powered devices generate no emissions during use and are relatively fast, but can only travel short distances. 
The bicycles also generate zero emissions during use, but the top speed and maximum range are highly variable depending on the user.
As battery technology continues to progress, electric modes of transport will become an increasingly viable option.

\subsection{Social Considerations}
Two of the major social considerations that need to be addressed with regards to electric transport are rider safety and cyber security. 
These two topics are carefully reviewed below.
\subsubsection{Rider Safety}
The most recent study of skateboard related trauma occurred in 2010 and focused on the last five years of data from the National Trauma Databank in the United States, further supporting the data with comparisons to other recent studies and analysis of datasets from outside sources \cite{Injury}. The study analyzed 2270 hospital patients admitted due to skateboarding related injuries. Of these, 8\% were under 10 years of age, 58\% were between 10 and 16 years of age, and the remaining 34\% were older than 16 \cite{Injury}. One statistic the researchers highlighted was the rate of incidence of traumatic brain injury among those admitted; in aforementioned age groups, these were 24.1\%, 32.6\%, and 45.5\% respectively\cite{Injury}. While the study concluded that ``helmet utilization and designated skateboard areas significantly reduce the incidence of serious head injuries'' \cite{Injury}, this also highlights the need to develop an electric personal transport vehicle that will minimize the chance of rider injury and, if possible, reduce the severity of unavoidable falls. As the study suggests, proper protective equipment will also be crucial for any user of the proposed product. 

For the final personal electric transport system, this means preventing the automated control from applying extreme acceleration that could cause the rider to be ejected from the vehicle. Similarly, it should keep the vehicle stable in all scenarios, preventing the rider from falling off during turns or at low speeds.

\subsubsection{Cybersecurity}
With any digital system in the 21st century, cybersecurity is a key concern as it has grown far beyond simply protecting information or resources against intruders \cite{cybersecurity}. As electric skateboards have risen in popularity, they have attracted the attention of major hacking conferences. The most notable example was a recent presentation at DEF CON, ``the world's longest running and largest underground hacking conference'' \cite{DEFCONsite}, where a group demonstrated techniques for jamming and overriding the wireless control signals used by three popular electric skateboard brands, allowing them to adjust speed, apply the brakes, and permanently disable the vehicles \cite{DEFCON,wiredArticle}. This has direct implications for rider safety since a sudden stop at high speeds could cause significant injury. 

These developments must be considered for a commercial personal electric transport product since some method of wireless speed control will likely be required to allow adequate control for the rider. The DEF CON presenters noted that proprietary RF (radio frequency) protocols were particularly easy to capture and replicate using SDR (software defined radio) \cite{Radio}\cite{DEFCON} but that more modern Bluetooth features could provide sufficient security to prevent hacking using current techniques \cite{DEFCON}. Possible technologies like these will be considered to provide a solution that minimizes potential cybersecurity threats.

\subsection{Environmental Considerations}
While electric transport greatly reduces the greenhouse gases produced during operation, there is a significant environmental impact associated with the end of life cycle disassembly. The battery technologies implemented in electronic transport are analyzed below.
\subsection{Battery Technologies}
Modern electronic devices generally rely on rechargeable lithium-ion or lithium-polymer batteries due to their energy storage density and long product lives \cite{BatteryRecharge}. A life cycle analysis of these types of batteries revealed a high cost to the environment and to human health; high lead content (averaging 6.29 mg/L \cite{BatteryRecharge}) and cobalt content (averaging 163544 mg/kg \cite{BatteryRecharge}) both cause them to be classified as hazardous according to U.S. federal regulations \cite{BatteryRecharge}. There are risks of  resource depletion, detrimental effects to human health, and ecotoxicity associated with these battery technologies \cite{BatteryRecharge} which will have to be carefully considered when developing a solution. The potential and feasibility of next generation energy storage technologies like graphene batteries \cite{Graphene} will also be analyzed.

Batteries in personal electric transportation devices became a topic of conversation when the US government recalled over 500,000 two wheeled, self balancing scooters in 2016 due to a risk of batteries sparking, catching fire and exploding, causing at least 18 injuries \cite{CBCArticle}. This brings a clear social impact as well since there is a safety risk associated with low cost batteries manufactured in China\cite{CBCArticle}.

