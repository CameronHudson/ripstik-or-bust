\section{Engineering Considerations}
\subsection{Legal Considerations}

\subsubsection{Regulatory Considerations}

\paragraph{Street Legality}\mbox{}\\
Many jurisdictions have laws and regulations pertaining to the operation of personal transport vehicles, both electric and human-powered. 
Toronto and Fredricton are among many Canadian cities which have laws prohibiting the use of skateboards on public streets. \cite{TOLaws} 
With regards to motorized transport, California had a long-standing ban on motorized skateboards which was recently abolished in the passing of Assembly Bill 2054 in 2015. \cite{OCLaws} \cite{WSJLaws}
Recently, the emergence of two-wheeled, self balancing electric scooters prompted the development of new laws and regulations. 
These devices are banned from airlines in Canada and the USA and their usage is heavily restricted or banned in regions including Toronto, Vancouver, New York City, and the state of California. \cite{leetboard}
It is imperative that a personal electric transport vehicle be compliant with relevant regulations and laws in its area of operation.


\paragraph{Intellectual Property}\mbox{}\\
In order to ensure that a personal electric transport system is commercially feasible, it is critical that all the appropriate measures to protect intellectual property be undertaken. 
As the final product will involve the incorporation of mechanical control system to a pre-existing complex mechanical vehicle, the final product will qualify as a combinatory invention. 
A combinatory invention involves combining two or more pre-existing technologies in order to develop a new technology. 
Combinatory inventions are among the most common patents; between the years 1790 and 2010, 77\% of all patents granted involved the combination of at least two pre-existing technologies. \cite{COMB} 
A patent shall not be issued, under United States patent law, if \emph{“the subject matter as a whole would have been obvious at the time the invention was made to a person having ordinary skill in the art to which said subject matter pertains. 
Patentability shall not be negatived by the manner in which the invention was made.}" \cite{NonObv} 
\\
\\
A 2007 United States Supreme Court ruling in “KSR International Co. v. Teleflex Inc." established legal precedent for “non-obviousness" in combinatory inventions. 
Teleflex Inc. sued KSR International Co. for patent infringement pertaining to a KSR product which Teleflex claimed infringed on its patent for an “Adjustable Pedal Assembly with Electronic Throttle Control."
KSR argued that Teleflex's claim was invalid as the action was obvious. 
The United States Supreme Court ruled in favour of KSR International Co.,  
ruling that Teleflex's invention was not patentable as the combination of the two adjustable pedal assembly and the electronic throttle control was indeed obvious. 
\\
\\
In developing novel methods of personal electric transport, it is imperative that the designers consider existing patent laws and cases in order to ensure that their products can be protected.

\subsection{Professional Practice Considerations and Ethics}

Professional Engineers are bound to a Code of Ethics dictating how they must behave. Section 77 of the Professional Engineers Ontario Code of Ethics states that \cite{PEO}: \emph{“it is the duty of a practitioner to the public, to the practitioner's employer, to the practitioner's clients, to other licensed engineers of the practitioner's profession, and to the practitioner to act at all times with,
\renewcommand{\labelenumi}{\roman{enumi}}
\begin{enumerate}
	\item fairness and loyalty to the practitioner's associates, employers, clients, subordinates and employees;
	\item fidelity to public needs;
	\item devotion to high ideals of personal honour and professional integrity;
	\item knowledge of developments in the area of professional engineering relevant to any services that are undertaken; and
	\item competence in the performance of any professional engineering services that are undertaken."
\end{enumerate}}
It is an engineer's obligation to the public to ensure that a personal electric transport device be safe for public use. 
It is imperative that the control system be designed to ensure safety in the device's operation, but also minimize the consequences of negligent use of the device. 
All engineers involved in the design and implementation of the control system must take all proper precautions to ensure that the system satisfy these constraints. 
The engineer must act in good faith and uphold the highest degree of professional integrity. 
\subsection{Economic Analysis}
\subsubsection{Cost Breakdown}
In order to determine the economic feasibility of an electric transport device, it was compared to traditional gas and human powered transportation devices.
An analysis was conducted based on three different metrics: initial cost, maintenance cost and fuel cost.
The electric powered devices used in the analysis were the \textit{OneWheel} and the \textit{Boosted Boards} electric longboard.
The Gas powered device used in the analysis was a \textit{2017 Yamaha Zuma 50F}.
The human powered device used in the analysis was a \textit{Raleigh Furley} bicycle.
\par
Initial costs represented upfront costs associated with purchasing each of the devices.
Maintenance costs represented the necessary costs for spare parts, and yearly upkeep of each device. 
Fuel costs covered the costs associated with the energy demand of the devices.	
The economic analysis for each method was conducted over a span of five years. The results can be seen in Table \ref{table:econ}.

\begin{table}[ht]
	\caption{Comparison of electric, gas, and human powered transportation devices over five years (all values in USD)}
	\centering
	\def\arraystretch{1.5}
	\begin{tabular}{|c| c| c| c| c|}
		\hline\hline
		\textbf{Method} & \textbf{Device}	& \textbf{Initial Cost (\$)} & \begin{tabular}{@{}c@{}} \textbf{Maintenance}\\ \textbf{Cost (\$)} \end{tabular} & \begin{tabular}{@{}c@{}} \textbf{Fuel} \\ \textbf{Cost (\$)} \end{tabular} \\ 
		\hline
		\multirow{2}{*}{Electric Powered} & OneWheel & 1,299 - 1,499 & 340 & 25 \\
		\cline{2-5}
		& Boosted Boards & 1,299 - 1,499 & 359 & 25\\
		\hline
		Gas Powered & Yamaha Zuma 50F & 2,599 & 371.40 & 155\\ 
		\hline
		Human Powered & Raleigh Furley & 980 & 969.90 & 0\\[0.1ex]	
		\hline
	\end{tabular}
	\label{table:econ}
\end{table}

The initial cost for the \textit{OneWheel} is US\$1299.00 for the base model, and US\$1499.00 for the Plus model \cite{wheelcost}.	
The initial cost for the \textit{Boosted Boards} electric long board is US\$1299.00 for the base model, and US\$1499.00 for the plus model \cite{boardcost}.
The initial cost for the \textit{Yamaha Zuma 50F} is US\$2,599.00 for the base model \cite{Yamaha}.
The initial cost for the \textit{Raleigh Furley} is US\$980.00 \cite{Raleigh}. 
The \textit{Raleigh Furley} was selected as it was ranked one of the top commuter bicycles for 2016 \cite{BikeMagazine}.
\par
The maintenance cost for the \textit{OneWheel} totals US\$340.00, consisting of one replacement charger (US\$125.00) and one tune up and reload pack (US\$215.00) \cite{wheelcost}.
The tune up and reload pack consists of a 17 point inspection, motor, battery health, hardware and firmware assessment, new footpads, new bumpers, and a new tire \cite{wheelcost}.
The maintenance cost for the \textit{Boosted Boards} electric long board totals US\$359.00, consisting of replacement parts for each of the key components on the board. This includes US\$105.00 for a full set of replacement wheels, US\$100.00 for a replacement remote, US\$79.00 for a replacement charger, US\$50.00 for a bearing service kit, and US\$25.00 for a motor belt service kit \cite{boardcost}.
The maintenance cost for the \textit{Yamaha Zuma 50F} consists of a maintenance kit with lubricant, oil filters, fuel filters, drain plugs, and a disposable funnel \cite{YamahaMaintenance}. 
The maintenance kit costs US\$74.28 and is required annually, meaning the five-year maintenance cost of the vehicle is US\$371.40.
The maintenance cost for the \textit{Raleigh Furley} consists of a tune up and drivetrain clean, along with two new tires each year. 
The tune up and drivetrain clean cost US\$150.00 per year, totalling US\$750.00 dollars over a five-year lifespan\cite{bikerepair}. 
The replacement tires cost US\$44.00 per year, and total to US\$220.00 \cite{CanadianTire}.
\par
The fuel costs for the \textit{OneWheel} and \textit{Boosted Board} were treated equally, assuming that they both use the same battery type. With the specified battery, the average yearly charging cost is US\$5.00, assuming a travelling distance 2000 miles per year \cite{boostedkickstart}. 
This yields a total cost of US\$25.00.
The fuel costs for the \textit{Yamaha Zuma 50F} are calculated assuming the same yearly travel distance of 2000 miles.
The Yamaha has a fuel tank that can hold 1.2 gallons of fuel, and a fuel efficiency of 132 miles per gallon \cite{Yamaha}.
The gas price was determined by assuming that the refills occur in New York State, where fuel is sold at a price of approximately US\$2.448 per gallon \cite{gasprice}.
With the information provided, the average fuel cost is US\$31.00 per year, totalling US\$155.00 over five years.
The fuel cost associated with a bicycle are zero, as it relies only on human force for operation.
\par
From a purely monetary standpoint, electric powered transportation vehicles are the cheapest option.

\subsubsection{Usage Characteristics}
Performance metrics need to be considered when comparing devices. These include the range of the device, top speed of the device, and environmental impact of using the device.
The range of the device is the total distance the device is able to travel until the fuel source runs out.
The top speed of the device is the maximum speed that the device can achieve, measured in kilometres per hour.
The environmental impact of using the device considers the emissions generated during vehicle operation.
The three factors were measured for each device and compiled in Table \ref{table:usage}.

\begin{table}[ht]
	\caption{Comparison of the range, top speed, and emissions for the transportation devices}
	\centering
	\def\arraystretch{1.3}
	\begin{tabular}{|c| c| c| c| c|}
		\hline\hline
		\textbf{Method} & \textbf{Device}	& \begin{tabular}{@{}c@{}} \textbf{Maximum} \\ \textbf{Range (km)} \end{tabular} & \begin{tabular}{@{}c@{}} \textbf{Top Speed}\\ \textbf{(km/h)}\end{tabular} & \textbf{Emissions}  \\ 
		\hline
		\multirow{2}{*}{Electric Powered} & OneWheel & 10-11 & 15-24 & 0 \\
		\cline{2-5}
		& Boosted Board & 10-19 & 35 & 0\\
		\hline
		Gas Powered & Yamaha Zuma 50F & 253 & 68 & \begin{tabular}{@{}c@{}} 1.0 g/km HC\\ 12 g/km $CO_{2}$\end{tabular}\\ 
		\hline
		Human Powered & Raleigh Furley Bicycle & 16 & 18-19 & 0\\[0.1ex]	
		\hline
	\end{tabular}
\label{table:usage}
\end{table}

The maximum range for the \textit{OneWheel} is 10-11 km \cite{wheelcost}.
The maximum range for the \textit{Boosted Boards} electric longboard is 10 km for the standard package, and 19 km for the plus package \cite{boardcost}.
Given a fuel efficiency of 132 miles per gallon and a tank size of 1.2 gallons, the maximum range of the \textit{Yamaha Zuma 50F} is 253 kilometres.
The average bicycle commuter travels a distance of 16 kilometres to and from their destination \cite{BikePaper}. However, this distance can vary based on the distance the rider needs to travel, their physical fitness level and the nature of the terrain and topology of the region.
\par
The top speed of the \textit{OneWheel} is between 15-24 kilometres per hour \cite{wheelcost}.
The top speed of the \textit{Boosted Boards} electric longboard is 45 kilometres per hour \cite{boardcost}.
The top speed of the \textit{Yamaha Zuma 50F} is 68 kilometres per hour \cite{Yamaha}.
The top speed of the \textit{Raleigh Furley} is dependant on the rider. Typical speeds for bicycle riders are in the range of 18-19 kilometres per hour \cite{bikespeed}.
\par
The only device that generates any environmental waste associated with its operation is the \textit{Yamaha Zuma 50F}. The average gas scooter emits approximately 1.0 gram/km of hydrocarbons and 12 grams/km of carbon dioxide \cite{emissions}. Note however that emissions associated in the production of the energy were omitted in this analysis. 
\par
A significant consideration in selecting a device is the trade-off between environmental impact and convenience.
The \textit{Yamaha Zuma 50F} has a significantly higher top speed and maximum range when compared to the other devices. 
However, it has a considerable environmental impact associated with its operation. Meanwhile, the electric powered devices generate no emissions during use and are relatively fast, but have limited range. 
The bicycles also generate zero emissions during use, but the top speed and maximum range are highly variable depending on the user.
Additionally, bicycles require considerably more exertion from the operation. 
As battery technology continues to develop, electric modes of transport will become an increasingly viable option as their range will continue to improve.

\subsection{Social Considerations}
Two of the major social considerations that need to be addressed with regards to electric transport are rider safety and cyber security. 
These two topics are carefully reviewed below.
\subsubsection{Rider Safety}
A recent study of skateboard related trauma occurred in 2010 and focused on the last five years of data from the National Trauma Databank in the United States, further supporting the data with comparisons to other recent studies and analysis of datasets from outside sources \cite{Injury}. The study analyzed 2270 hospital patients admitted due to skateboarding related injuries. Of these, 8\% were under 10 years of age, 58\% were between 10 and 16 years of age, and the remaining 34\% were older than 16 \cite{Injury}. One statistic the researchers highlighted was the rate of incidence of traumatic brain injury among those admitted; in aforementioned age groups, these were 24.1\%, 32.6\%, and 45.5\% respectively\cite{Injury}. While the study concluded that ``helmet utilization and designated skateboard areas significantly reduce the incidence of serious head injuries'' \cite{Injury}, this also highlights the need to develop an electric personal transport vehicle that will minimize the chance of rider injury and, if possible, reduce the severity of unavoidable falls. As the study suggests, proper protective equipment will also be crucial for any user of the proposed product. 

For the final personal electric transport system, this means preventing the automated control from applying extreme acceleration that could cause the rider to be ejected from the vehicle. Similarly, it should keep the vehicle stable in all scenarios, preventing the rider from falling off during turns or at low speeds.

\subsubsection{Cybersecurity}
With any digital system in the 21st century, cybersecurity is a key concern as it has grown far beyond simply protecting information or resources against intruders \cite{cybersecurity}. As electric skateboards have risen in popularity, they have attracted the attention of major hacking conferences. The most notable example was a recent presentation at DEF CON, ``the world's longest running and largest underground hacking conference'' \cite{DEFCONsite}, where a group demonstrated techniques for jamming and overriding the wireless control signals used by three popular electric skateboard brands, allowing them to adjust speed, apply the brakes, and permanently disable the vehicles \cite{DEFCON,wiredArticle}. This has direct implications for rider safety since a sudden stop at high speeds could cause significant injury. 

These developments must be considered for a commercial personal electric transport product since some method of wireless speed control will likely be required to allow adequate control for the rider. The DEF CON presenters noted that proprietary RF (radio frequency) protocols were particularly easy to capture and replicate using SDR (software defined radio) \cite{Radio}\cite{DEFCON} but that more modern Bluetooth features could provide sufficient security to prevent hacking using current techniques \cite{DEFCON}. Possible technologies like these will be considered to provide a solution that minimizes potential cybersecurity threats.

\subsection{Environmental Considerations}
While electric transport greatly reduces the greenhouse gases produced during operation, there is a significant environmental impact associated with the end of life cycle disassembly. The battery technologies implemented in electronic transport are analyzed below.
\subsubsection{Battery Technologies}
Modern electronic devices generally rely on rechargeable lithium-ion or lithium-polymer batteries due to their energy storage density and long product lives \cite{BatteryRecharge}. A life cycle analysis of these types of batteries revealed a high cost to the environment and to human health; high lead content (averaging 6.29 mg/L \cite{BatteryRecharge}) and cobalt content (averaging 163544 mg/kg \cite{BatteryRecharge}) both cause them to be classified as hazardous according to U.S. federal regulations \cite{BatteryRecharge}. There are risks of resource depletion, detrimental effects to human health, and ecotoxicity associated with these battery technologies \cite{BatteryRecharge} which will have to be carefully considered when developing a solution. The potential and feasibility of next generation energy storage technologies like graphene batteries \cite{Graphene} should also be analyzed.

Batteries in personal electric transportation devices became a topic of conversation when the US government recalled over 500,000 two wheeled, self balancing scooters in 2016 due to a risk of batteries sparking, catching fire and exploding, causing at least 18 injuries \cite{CBCArticle}. This brings a clear social impact as well since there is a safety risk associated with low cost batteries manufactured in China \cite{CBCArticle}.

\section{Concessions in Design Implementation} \label{sec:Tradeoff}
There are many performance parameters of an electric personal transport vehicle that can be addressed through the design and implementation of a mechanical control system. 
These parameters include speed, handling and stability. 
However, not all parameters can be optimized through the tuning of the system; trade offs in system performance are required. 
There are both trade-offs in the control algorithm and the mechanical control system.

\subsection{Control Algorithm} 
In designing a control algorithm, the primary parameters of focus are: rise time, settling time, percent overshoot and steady-state error. \cite{LQRpar}
Rise time refers to the amount of time it takes for the system output to reach the desired output. 
In the context of the vehicle, a slow rise time may mean that the system is not able to stabilize a soon-to-fall rider before it is too late. 
Clearly, the rise time of the control algorithm is a priority in order for the personal electric transport device to be a viable, safe vehicle.
The settling time is also of significance to the performance of the vehicle. 
The settling time must be adequately small to ensure that the vehicle can stabilize in a practical setting. 
A lengthy settling time may result in the vehicle oscillating around a stable configuration, making it difficult to operate. 
System overshoot refers to the ability of the vehicle to return to a stable configuration in a direct manner. 
If the system overshoots the desired, stable configuration, it could result in the ejection of the operator from the vehicle. 
Clearly, system overshoot should be minimized if the control system will be used in a viable personal electric transport vehicle. 
Finally, steady-state error refers to the ability of the control system to accurately return to its stable configuration following a disturbance. 
A large steady-state error would mean that, in returning to the stable configuration following a disturbance, the system would not fully return to the original stable configuration. 
Instead, it would experience a permanent deviation from said configuration. 
The primary objective of the controller is to improve the safety and operation of the personal electric transport vehicle. 
If the vehicle is not able to accurately return to its stable configuration, it could compromise the ability of the control system to accomplish its objective.
\par
Clearly, all four tuning parameters are of significant consequence in the design of a control algorithm to govern the operation of a personal electric transport vehicle. 
However, it is not possible to maximize each parameter individually in the design of a control alogrithm. 
Therefore, it is imperative that the final control algorithm be optimized to maximize performance parameters of the vehicle, while ensuring that it is robust enough to handle a variety of conditions in a variety of environments. 
Primary areas of focus in the design and tuning of the control system will vary depending on the nature of the complex mechanical system in which they are designed to govern.

\subsection{Mechanical Control System}
The control algorithm will be implemented by a mechanical control system located on the vehicle. 
This control system will require a battery power source.
As aforementioned, there are significant environmental consequences associated the manufacture and disposal of batteries. 
Minimizing the energy requirements of the mechanical control system will reduce the size and strength of battery required, thereby reducing the environmental impact of the personal electric transport vehicle.  
However, minimizing the energy requirements of the system would require minimizing the gain of the control system. 
Minimizing the gain would negatively impact the performance of the system, specifically as it pertains to the ability of the device to quickly recover from unstable configurations. 
It is imperative that the system be able to sufficiently respond to such disturbances as this is a pivotal requirement of a personal electric transport vehicle. 
Subsequently, engineers designing the control algorithm must ensure that the system is robust enough to respond to environmental changes in accordance with their duty to protect the public as specified in the Professional Engineers Ontraio Code of Ethics. 
An engineer cannot, in good faith, approve an mechanical control system design which could endanger the rider. 
Furthermore, a larger energy source will result in greater range and top speed of the vehicle. 
If the proposed personal electric vehicle is to be a viable option for urban transportation, its range and speed must be sufficiently large enough to ensure that it can reach urban destinations in a reasonable amount of time.
These are two of the outlines of an effective modern transportation system specified in the introduction. 
A trade-off between environmental considerations and project performance, constraints and ethics requires careful consideration. 
A solution in which the battery unit is sufficiently powerful enough to adequately address the performance benchmarks of the vehicle should be sought. 