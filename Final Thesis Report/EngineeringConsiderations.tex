\section{Engineering Considerations}
\subsection{Legal Considerations}
\subsubsection{Regulatory Considerations}
\subsubsection{Professional Practice Considerations}
\subsection{Design Considerations}
%%Safety vs Power, cost vs power, battery life vs environmental considerations & power & cost, stability vs mobility, cost vs sustainability
\subsection{Ethics}
%Manufacturing, disposal, safety???, Marketing???
\subsection{Economic Analysis}
\subsubsection{Cost Breakdown}
In order to determine the economic feasibility of an electric transport device, it was compared to traditional gas and human powered transportation devices.
An analysis was conducted based on three different metrics; Initial cost, maintenance cost and fuel cost.
The electric powered devices used in the analysis were the OneWheel and Boosted board.
The Gas powered device used in the analysis was a 2017 Yamaha Zuma 50F.
The human powered device used in the analysis was a  Raleigh Furley bicycle.
\par
Initial cost looked at the upfront costs associated with purchasing each of the devices.
Maintenance cost looked at the necessary costs required for spare parts, and yearly upkeep of each device. 
Fuel costs looked at the cost required to power each of the devices.	
The economic analysis for each method was conducted over a span of five years, and the results can be seen in Table \ref{table:econ}.

\begin{table}[ht]
	\caption{Comparison of electric, gas, and human powered transportation devices over five years (all values in USD)}
	\centering
	\def\arraystretch{1.5}
	\begin{tabular}{|c| c| c| c| c|}
		\hline\hline
		\textbf{Method} & \textbf{Device}	& \textbf{Initial Cost (\$)} & \begin{tabular}{@{}c@{}} \textbf{Maintenance}\\ \textbf{Cost (\$)} \end{tabular} & \begin{tabular}{@{}c@{}} \textbf{Fuel} \\ \textbf{Cost (\$)} \end{tabular} \\ 
		\hline
		\multirow{2}{*}{Electric Powered} & OneWheel & 1,299 - 1,499 & 340 & 25 \\
		\cline{2-5}
		& Boosted Board & 1,299 - 1,499 & 359 & 25\\
		\hline
		Gas Powered & Yamaha Zuma 50F & 2,599 & 371.40 & 155\\ 
		\hline
		Human Powered & Raleigh Furley Bicycle & 980 & 969.90 & 0\\[0.1ex]	
		\hline
	\end{tabular}
	\label{table:econ}
\end{table}

The inital cost for the OneWheel was \$1299.00 (USD) for the base model, and \$1499.00 (USD) for the Plus model \cite{wheelcost}.	
The initial cost for the Boosted Board was \$1299.00 (USD) for the base model, and \$1499.00 (USD) for the plus model \cite{boardcost}.
The initial cost for the Yamaha Zuma 50F was \$2,599.00 (USD) for the base model \cite{Yamaha}.
The initial cost for the Raleigh Furley Bicycle was \$980.00 (USD) \cite{Raleigh}. 
The Raleigh was selected since it was ranked one of the top commuter bicycles for 2016 \cite{BikeMagazine}.
\par
The maintenance cost for the OneWheel totaled to \$340.00 (USD), and consisted of one replacement charger (\$125.00 (USD)), and one tune up and reload pack (\$215.00 (USD)) \cite{wheelcost}.
The tune up and reload pack consisted of a 17 point inspection, motor, battery health, hardware and firmware assessment, new footpads, new bumpers, and a new tire \cite{wheelcost}.
The maintenance cost for the boosted board totaled to \$359.00 (USD), and consisted of replacement parts for each of the key components on the board. This includes  \$105.00 (USD) for a full set of replacement wheels, \$100.00 (USD) for a replacement remote,  \$79.00 (USD) for a replacement charger, \$50.00 (USD) for a bearing service kit, and \$25.00 (USD) for a motor belt service kit \cite{boardcost}.
The maintenance cost for the Yamaha Zuma 50F consisted of a maintenance kit with lube, oil filters, fuel filters,drain plugs, and a disposable funnel \cite{YamahaMaintenance}. 
This led to a cost of \$74.28 (USD) per year, totaling to \$371.40 (USD).
The maintenancecost for the Raleigh Furley bicycle consisted of a tune up and drivetrain clean, along with two new tires each year. 
The tune up and drivetrain clean cost \$150.00 (USD) per year, totaling to \$750.00 dollars \cite{bikerepair}. 
The replacement tires cost \$44.00 (USD) per year, and totaled to \$220.00 (USD) \cite{CanadianTire}.
\par
The fuel costs for the OneWheel and Boosted Board were treated equally, assuming that they both use the same battery type. With the specified battery, the average yearly charging cost is \$5.00 (USD), assuming a travelling distance 2000 miles \cite{boostedkickstart}. 
This leads to a total cost of \$25.00 (USD).
The fuel costs for the Yamaha Zuma 50F were calculated assuming the same yearly travel distance of 2000 miles.
The Yamaha has a fuel tank that can hold 1.2 Gallons of fuel, and a fuel efficiency of 132 miles per gallon \cite{Yamaha}.
The gas price was selected by assuming that the refills are occuring in New York State, leading to a price of \$2.448 (USD) per gallon \cite{gasprice}.
With the information provided, the average fuel cost came to \$31.00 (USD) per year, totaling to \$155.00 (USD).
The fuel cost associated with a bicycle is zero, as it relies only on human force for operation.
\par
From a purely monetary standpoint, the electric powered transportation is the cheapest option.

\subsubsection{Usage Characteristics}
A few other factors need to be considered, such as the range of the device, top speed of the device, and environmental impact of using the device.
The range of the device will consider total distance until the fuel source runs out.
The top speed of the device will consider the maximum speed that the device can achieve in kilometers per hour.
The environmental impact of using the device will consider the emissions generated during vehicle operation.
The three factors were measured for each device and compiled in Table \ref{table:usage}.

\begin{table}[ht]
	\caption{Comparison of the range, top speed, and emissions for the transportation devices}
	\centering
	\def\arraystretch{1.3}
	\begin{tabular}{|c| c| c| c| c|}
		\hline\hline
		\textbf{Method} & \textbf{Device}	& \begin{tabular}{@{}c@{}} \textbf{Maximum} \\ \textbf{Range (km)} \end{tabular} & \begin{tabular}{@{}c@{}} \textbf{Top Speed}\\ \textbf{(km/h)}\end{tabular} & \textbf{Emissions}  \\ 
		\hline
		\multirow{2}{*}{Electric Powered} & OneWheel & 10-11 & 15-24 & 0 \\
		\cline{2-5}
		& Boosted Board & 10-19 & 35 & 0\\
		\hline
		Gas Powered & Yamaha Zuma 50F & 253 & 68 & \begin{tabular}{@{}c@{}} 1.0 g/km HC\\ 12 g/km $CO_{2}$\end{tabular}\\ 
		\hline
		Human Powered & Raleigh Furley Bicycle & 16 & 18-19 & 0\\[0.1ex]	
		\hline
	\end{tabular}
\label{table:usage}
\end{table}

The maximum range for the OneWheel is 10-11 km \cite{wheelcost}.
The maximum range for the Boosted Board is 10 km for the standard package, and 19 km for the plus package \cite{boardcost}.
Given a fuel efficiency of 132 miles per gallon and a tank size of 1.2 gallons, the maximum range of the Yamaha Zuma 50F is 253 kilometers.
The average bicycle commuter travels a distance of 16 kilometers to and from their destination \cite{BikePaper}, but this distance can vary based on the distance the rider needs to travel and their physical fitness level.
\par
The top speed of the OneWheel is between 15-24 kilometers per hour \cite{wheelcost}.
The top speed of the Boosted Board is 45 kilometers per hour \cite{boardcost}.
The top speed of the Yamaha Zuma 50F is 68 kilometers per hour \cite{Yamaha}.
The top speed of the Raleigh Furley Bicycle is dependant on the rider, but typical speed for bicycle riders are in the range of 18-19 kilometers per hour \cite{bikespeed}.
\par
The only device that generates any environmental waste is the Yamaha Zuma 50F. The average gas scooter emits approximately 1.0 gram/km of hydrocarbons and 12 grams/km of carbon dioxide \cite{emissions}.
\par
The key trade-off to consider when selecting a device is environmental impact versus convenience.
The Yamaha Zuma 50F has a significantly higher top speed and maximum range when compared to the other devices. 
However, it generates a significant amount of emissions. The electric powered devices generate no emissions during use and are relatively fast, but can only travel short distances. 
The bicycles also generate zero emissions during use, but the top speed and maximum range are highly variable depending on the user.
As batter technology continues to progress, electric modes of transport will become an increasingly viable option.