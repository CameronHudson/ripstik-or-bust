\documentclass[12pt,letterpaper]{article}
\usepackage{cite}
\usepackage{amsmath}
\usepackage{amsfonts}
\usepackage{array}
\usepackage{dsfont}
\usepackage{amssymb}
\usepackage{amsthm}
\usepackage{bbold}
\usepackage{fullpage}
\usepackage{mathtools}
\usepackage{enumitem}
\usepackage{mathrsfs}
\usepackage[margin=0.9 in]{geometry}
\usepackage{hyperref}
\usepackage{graphicx}
\usepackage{gensymb}
\usepackage{xcolor,colortbl}
\usepackage[format=plain,
labelfont={bf,it},
textfont={it}]{caption}
\usepackage{float}


\newcommand*{\SignatureAndDate}[1]{%
	\par\noindent\makebox[2.5in]{\hrulefill} \hfill\makebox[2.0in]{\hrulefill}%
	\par\noindent\makebox[2.5in][l]{#1}      \hfill\makebox[2.0in][l]{Date}%
}%
\newcolumntype{L}{>{\centering\arraybackslash}m{2cm}}
\newcolumntype{P}{>{\centering\arraybackslash}m{3cm}}
\newcolumntype{Q}{>{\centering\arraybackslash}m{4cm}}
\setlength{\parindent}{0em}

\allowdisplaybreaks

\newcommand{\R}{\mathds{R}}
\newcommand{\Z}{\mathds{Z}}
\newcommand{\Rplus}{\mathds{R}_{> 0}}
\newcommand{\Zplus}{\mathds{Z}_{\geq 0}}
\newcommand{\F}{\mathds{F}}
\newcommand{\N}{\mathds{N}}
\newcommand{\T}{\mathds{T}}
\newcommand{\s}{\mathds{S}}
\newcommand{\C}{\mathds{C}}
\newcommand{\CDFT}{\mathscr{F}_{CD}} %Fourier transform
\newcommand{\ip}[2]{\langle #1, #2\rangle}


\setlength{\parskip}{0.5em}


\makeatletter
\newsavebox\myboxA
\newsavebox\myboxB
\newlength\mylenA

\begin{document}
	
\section{Trade-Offs}
There are many performance parameters of an electric personal transport vehicle that can be addressed through the design and implementation of a mechanical control system. 
These parameters include speed, handling and stability. 
However, not all parameters can be optimized through the tuning of the system; trade off in system performance is required. 
There are both trade-offs in the control algorithm and the mechanical control system.

\subsection{Control Algorithm} 
In designing a control algorithm, the primary parameters of focus are: rise time, settling time, percent overshoot and steady-state error. 
Rise time refers to the amount of time it takes for the system output to reach the desired output. 
In the context of the vehicle, a slow rise time may mean that the system is not able to stabilize a soon-to-fall rider before it is too late. 
Clearly, the rise time of the control algorithm is a priority in order for the personal electric transport device to be a viable, safe vehicle.
The settling time is also of significance to the performance of the vehicle. 
The settling time must be adequately small to ensure that the vehicle can stabalize in a practical setting. 
A lengthy settling time may result in the vehicle oscillating around a stable configuration, making it difficult to operate. 
System overshoot 

\subsection{Mechanical Control System}
The control algorithm will be implemented by a mechanical control system located on the vehicle. 
This control system will require a battery power source.
As aforementioned, there are significant environmental consequences associated the manufacture and disposal of batteries. 
Minimizing the energy requirements of the mechanical control system will reduce the size and strength of battery required, thereby reducing the environmental impact of the personal electric transport vehicle.  
However, minimizing the energy requirements of the system would require minimizing the gain of the control system. 
Minimizing the gain would negatively impact the performance of the system, specifically as it pertains to the ability of the device to quickly recover from unstable configurations. 
It is imperative that the system be able to sufficiently respond to such disturbances as this is a pivotal requirement of a personal electric transport vehicle. 
Subsequently, engineers designing the control algorithm must ensure that the system is robust enough to respond to environmental changes in accordance with their duty to protect the public as specified in the Professional Engineers Ontraio Code of Ethics. 
An engineer cannot, in good faith, approve an mechanical control system design which could endanger the rider. 
Furthermore, a larger energy source will result in greater range and top speed of the vehicle. 
If the proposed personal electric vehicle is to be a viable option for urban transportation, its range and speed must be sufficiently large enough to ensure that it can reach urban destinations in a reasonable amount of time.
These are two of the outlines of an effective modern transportation system specified in the introduction. 
A trade-off between environmental considerations and project performance, constraints and ethics requires careful consideration. 
A solution in which the battery unit is sufficiently powerful enough to adequately address the performance benchmarks of the vehicle should be sought. 


\newpage
\bibliography{Bibliography}{}
\bibliographystyle{unsrt}

\end{document} 