\documentclass[12pt,letterpaper]{article}
\usepackage{cite}
\usepackage{amsmath}
\usepackage{amsfonts}
\usepackage{array}
\usepackage{dsfont}
\usepackage{amssymb}
\usepackage{amsthm}
\usepackage{bbold}
\usepackage{fullpage}
\usepackage{mathtools}
\usepackage{enumitem}
\usepackage{mathrsfs}
\usepackage[margin=0.9 in]{geometry}
\usepackage{hyperref}
\usepackage{graphicx}
\usepackage{gensymb}
\usepackage{xcolor,colortbl}
\usepackage[format=plain,
labelfont={bf,it},
textfont={it}]{caption}
\usepackage{float}


\newcommand*{\SignatureAndDate}[1]{%
	\par\noindent\makebox[2.5in]{\hrulefill} \hfill\makebox[2.0in]{\hrulefill}%
	\par\noindent\makebox[2.5in][l]{#1}      \hfill\makebox[2.0in][l]{Date}%
}%
\newcolumntype{L}{>{\centering\arraybackslash}m{2cm}}
\newcolumntype{P}{>{\centering\arraybackslash}m{3cm}}
\newcolumntype{Q}{>{\centering\arraybackslash}m{4cm}}
\setlength{\parindent}{0em}

\allowdisplaybreaks

\newcommand{\R}{\mathds{R}}
\newcommand{\Z}{\mathds{Z}}
\newcommand{\Rplus}{\mathds{R}_{> 0}}
\newcommand{\Zplus}{\mathds{Z}_{\geq 0}}
\newcommand{\F}{\mathds{F}}
\newcommand{\N}{\mathds{N}}
\newcommand{\T}{\mathds{T}}
\newcommand{\s}{\mathds{S}}
\newcommand{\C}{\mathds{C}}
\newcommand{\CDFT}{\mathscr{F}_{CD}} %Fourier transform
\newcommand{\ip}[2]{\langle #1, #2\rangle}


\setlength{\parskip}{0.5em}


\makeatletter
\newsavebox\myboxA
\newsavebox\myboxB
\newlength\mylenA

\begin{document}
	
\section{Trade-Offs}
There are many performance parameters of an electric personal transport vehicle that can be addressed through the design and implementation of a mechanical control system. 
These parameters include speed, handling and stability. 
However, not all parameters can be optimized through the tuning of the system; trade off in system performance is required. 
There are both trade-offs in the control algorithm and the mechanical control system.

\subsection{Control Algorithm} 
In designing a control algorithm, the primary parameters of focus are: rise time, settling time, percent overshoot and steady-state error. 
Rise time refers to the amount of time it takes for the system output to reach the desired output. 
In the context of the vehicle, a slow rise time may mean that the system is not able to stabilize a soon-to-fall rider before it is too late. 
Clearly, the rise time of the control algorithm is a priority in order for the personal electric transport device to be a viable, safe vehicle.
The settling time is also of significance to the performance of the vehicle. 
The settling time must be adequately small to ensure that the vehicle can stabalize in a practical setting. 
A lengthy settling time may result in the vehicle oscillating around a stable configuration, making it difficult to operate. 
System overshoot 

\subsection{Mechanical Control System}
The control algorithm will be implemented by a mechanical control system located on the vehicle. 
This control system will require a power source and a housing system. 


\newpage
\bibliography{Bibliography}{}
\bibliographystyle{unsrt}

\end{document} 